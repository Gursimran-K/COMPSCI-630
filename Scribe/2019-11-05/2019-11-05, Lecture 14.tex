\documentclass[twoside]{article}
\setlength{\oddsidemargin}{0.25 in}
\setlength{\evensidemargin}{-0.25 in}
\setlength{\topmargin}{-0.6 in}
\setlength{\textwidth}{6.5 in}
\setlength{\textheight}{8.5 in}
\setlength{\headsep}{0.75 in}
\setlength{\parindent}{0 in}
\setlength{\parskip}{0.1 in}

\usepackage{graphicx}
\usepackage{listings}
\usepackage{url}
\usepackage{amsmath}
\usepackage{amsfonts}

%
% The following commands sets up the lecnum (lecture number)
% counter and make various numbering schemes work relative
% to the lecture number.
%
\newcounter{lecnum}
\renewcommand{\thepage}{\thelecnum-\arabic{page}}
\renewcommand{\thesection}{\thelecnum.\arabic{section}}
\renewcommand{\theequation}{\thelecnum.\arabic{equation}}
\renewcommand{\thefigure}{\thelecnum.\arabic{figure}}
\renewcommand{\thetable}{\thelecnum.\arabic{table}}
\newcommand{\dnl}{\mbox{}\par}

%
% The following macro is used to generate the header.
%
\newcommand{\lecture}[4]{
  \pagestyle{myheadings}
  \thispagestyle{plain}
  \newpage
  \setcounter{lecnum}{#1}
  \setcounter{page}{1}
  \noindent
  \begin{center}
  \framebox{
     \vbox{\vspace{2mm}
   \hbox to 6.28in { {\bf CMPSCI~630~~~ Systems
                       \hfill Fall 2019} }
      \vspace{4mm}
      \hbox to 6.28in { {\Large \hfill Lecture #1  \hfill} }
%       \hbox to 6.28in { {\Large \hfill Lecture #1: #2  \hfill} }
      \vspace{2mm}
      \hbox to 6.28in { {\it Lecturer: #3 \hfill Scribe: #4} }
     \vspace{2mm}}
  }
  \end{center}
  \markboth{Lecture #1: #2}{Lecture #1: #2}
  \vspace*{4mm}
}

%
% Convention for citations is authors' initials followed by the year.
% For example, to cite a paper by Leighton and Maggs you would type
% \cite{LM89}, and to cite a paper by Strassen you would type \cite{S69}.
% (To avoid bibliography problems, for now we redefine the \cite command.)
%
\renewcommand{\cite}[1]{[#1]}

% \input{epsf}

%Use this command for a figure; it puts a figure in wherever you want it.
%usage: \fig{NUMBER}{FIGURE-SIZE}{CAPTION}{FILENAME}
\newcommand{\fig}[4]{
           \vspace{0.2 in}
           \setlength{\epsfxsize}{#2}
           \centerline{\epsfbox{#4}}
           \begin{center}
           Figure \thelecnum.#1:~#3
           \end{center}
   }

% Use these for theorems, lemmas, proofs, etc.
\newtheorem{theorem}{Theorem}[lecnum]
\newtheorem{lemma}[theorem]{Lemma}
\newtheorem{proposition}[theorem]{Proposition}
\newtheorem{claim}[theorem]{Claim}
\newtheorem{corollary}[theorem]{Corollary}
\newtheorem{definition}[theorem]{Definition}
\newenvironment{proof}{{\bf Proof:}}{\hfill\rule{2mm}{2mm}}

% Some useful equation alignment commands, borrowed from TeX
\makeatletter
\def\eqalign#1{\,\vcenter{\openup\jot\m@th
 \ialign{\strut\hfil$\displaystyle{##}$&$\displaystyle{{}##}$\hfil
     \crcr#1\crcr}}\,}
\def\eqalignno#1{\displ@y \tabskip\@centering
 \halign to\displaywidth{\hfil$\displaystyle{##}$\tabskip\z@skip
  &$\displaystyle{{}##}$\hfil\tabskip\@centering
  &\llap{$##$}\tabskip\z@skip\crcr
  #1\crcr}}
\def\leqalignno#1{\displ@y \tabskip\@centering
 \halign to\displaywidth{\hfil$\displaystyle{##}$\tabskip\z@skip
  &$\displaystyle{{}##}$\hfil\tabskip\@centering
  &\kern-\displaywidth\rlap{$##$}\tabskip\displaywidth\crcr
  #1\crcr}}
\makeatother

% **** IF YOU WANT TO DEFINE ADDITIONAL MACROS FOR YOURSELF, PUT THEM HERE:



% Some general latex examples and examples making use of the
% macros follow.

\begin{document}

%FILL IN THE RIGHT INFO.
%\lecture{**LECTURE-NUMBER**}{**DATE**}{**LECTURER**}{**SCRIBE**}
\lecture{14}{October 17}{Emery Berger}{Adeel Hassan}

\section{Load management}

\subsection{Little's Law}
\begin{align}
    L = \lambda * w
\end{align}
where\\
\begin{align}
    L &= \text{number of customers waiting in line or being served}\\
    \lambda &= \text{arrival rate}\\
    w &= \text{service time}
\end{align}

The law holds under the assumption that the system is \textbf{backlogged} i.e. there are always some customers waiting in line.

\subsection{Load shedding}
This is a form of load management where you turn away customers to avoid degradation of service or overload. This strategy is also a form of \textbf{admission control}.
\begin{itemize}
    \item \textbf{Example:} a bouncer outside a bar refusing to let more customers in.
\end{itemize}

Degradation of service can be further sub-categorized as:
\begin{itemize}
    \item Degradation of \textbf{Quality of Service (QoS)}
    \item \textbf{Denial of Service (DoS)}
\end{itemize}

\subsection{Multi-threading}
Two strategies:
\begin{enumerate}
    \item Spawn a new thread for every new customer
        \begin{itemize}
            \item Bad if too many customers show up
        \end{itemize}
    \item Maintain a thread pool. When a new customer arrives assign a thread (if available) from the pool. If all threads in the pool are busy, the customer will simply have to wait.
    \begin{itemize}
        \item Acts as a sort of admission control mechanism, since the system cannot take on more load than it can handle.
    \end{itemize}
\end{enumerate}

\section{Cryptography}
\textbf{Crypt} is Greek for "secret".

\subsection{Security through obscurity}
A security strategy that involves hiding the details of the system implementation as a form of security mechanism.

A \textbf{VERY BAD} idea. Considered to be no security at all.

\subsection{Steganography}
Hiding a message inside another message. The "key" is the embedding method.

\textbf{Example:} Hiding a message in the least significant bits of an image. The change in pixel color values will be too small for the human eye to notice.

\subsection{Caesar Cipher}
"Shift" all characters in the message by a fixed amount so that they map to different characters. The key is the shift amount. For example, if the key is 5, the characters map as follows:
\begin{align}
    A &\rightarrow E\\
    B &\rightarrow F\\
    C &\rightarrow G\\
    &\;\;\vdots \notag
\end{align}

Not a very strong encryption scheme. Requires max 26 tries to break.

\subsection{A better version}
Create a unique mapping for each character. The key, in this case, is all the character mappings.

There are $26!$ possible mappings, so a brute force decryption attempt will require $26! = 403291461126605635584000000$ attempt, which is impractical.

A cleverer way to break this kind of encryption is to use \textbf{frequency analysis}.
\begin{itemize}
    \item The most frequent letters in English are known to be (in decreasing order):  ETAOINSHRDLU. One can try to match the most frequent letters in the cipher text to these.
    \item If some random letter appears as a one-letter word in the cipher text, it is very likely to be the mapping for \textit{a} or \textit{i}.
    \item Similarly if a 3-letter word appears very often, it is likely to be \textit{the}, which gives you the mappings for \textit{t}, \textit{h}, and \textit{e}.
\end{itemize}

Frequency analysis works really well and allows such ciphers to be easily broken.

\subsection{The Perfect cipher}
There exists an encryption scheme known to be theoretically unbreakable: the \textbf{one-time pad}.

\textbf {How it works}: create a unique mapping for each character in the message. Use it once for that message and never again. The key length is equal to the message length.

(This is distinct from the previous cipher because here the same character will have a different mapping for each position that it occurs in the message.)

In terms of binary strings, encryption is performed by, XOR'ing the message with the key. The encrypted message can be XOR'ed with the key again to decrypt it.
\begin{align}
    msg &= 01101101\;\;01110011\;\;01100111\\
    key &= 01101011\;\;01100101\;\;01111001\\
    E(msg, key) &= msg \otimes key\\
    &= 00000110\;\;00010110\;\;00011110\\
    msg &= E(msg, key) \otimes key\\
    &= 01101101\;\;01110011\;\;01100111
\end{align}

The reason this is not used for everything is that it is impractical. For every message, you have to generate and safely transmit the key to the recipient.

\subsection{The Enigma machine}
Encryption machine used by the German during WW2. Used a complicated encryption scheme that defies frequency analysis. The character mappings change with every key stroke.

It was broken by Polish and British mathematicians using, among other things, a \textbf{known plaintext attack}. This involves somehow figuring out what the plaintext for a part of the ciphertext is, and using that to decrypt the reset of the message.

\textbf{Anecdote}: during the first gulf war, the Iraqis were using encrypted walkie-talkies bought from the British that did not actually work, resulting in their communication being aired out in the plain. This allowed the American forces to eavesdrop easily and gain a significant advantage.

\section{Encryption/Decryption}
\textbf{Goal}: encrypted message should look like random noise. 

\textbf{Goal}: The best an attacker should be able to do is brute force.

One way to do this is to make the key long. The key length can be used to obtain a probabilistic guarantee for the time to decrypt by brute force:

\begin{align}
    \mathbb{E}[\text{time to decrypt}] = \frac{1}{2} \times |\text{key space}|
\end{align}

For example, Stripe uses a key length of 2048 bits. $\Rightarrow |\text{key space}| = 2^{2048}$.

\subsection{Asymmetric key encryption}
In a paper, Diffie and Helman proposed an encryption scheme based on a \textbf{one-way function} AKA a function that is easy to compute but whose inverse is hard to compute.
    \begin{align}
        f(\alpha) &= \beta & \text{"easy"}\\
        f'(\beta) &= \alpha & \text{"hard"}
    \end{align}
\subsubsection{RSA}
Uses the prime factoring as the one-way function. In this case, the easy direction is the multiplication of two large prime numbers and the hard direction is factoring that large number into its prime factors.

Prime factorization is not NP-complete. We don't know exactly how hard it is.

key = $(e, d, n)$. $n=pq$. $e$ computed from $d$. $d$ relatively prime to ($p-1$, $q-1$).

$e$ is public. $d$ is private.

\textbf{Signing and encryption:}

\begin{align}
    d_{Bob}(m) &= s & \text{(sign - Bob encrypts with his private key)}\\
    e_{Alice}(s) &= s_{Alice} & \text{(encrypt - Bob encrypts with Alice's public key)}\\
    d_{Alice}(s_{Alice}) &= s & \text{(decrypt - Alice decrypts with her private key)}\\
    e_{Bob}(s) &= m & \text{(decrypt - Alice decrypts with Bob's public key to obtain the message)}
\end{align}

RSA is used in TLS.

\textbf{Where to get the public key?}
\begin{itemize}
    \item PKI - Public Key Infrastructure
    \item Root Certificate Authorities
    \item Who to trust? \textit{Quis custodiet ipsos custodes}. "Who guards the guardians?" Turtules all the way down
\end{itemize}

\section{Fault Tolerance}

\begin{align}
    \text{availability} &= \frac{\text{MTBF}}{\text{MTBF}+\text{MTTR}}\\\\
    \text{MTBF} &= \text{mean time between failures}\\
    \text{MTBF} &= \text{mean time to recover}
\end{align}

\textbf{ROC - Recovery Oriented Programming}
\begin{itemize}
    \item reduce MTTR
\end{itemize}

\textbf{Fail safe vs fail stop}
\begin{itemize}
    \item rollback/undo error vs stop on error
\item \begin{tabular}{c|c}
     rollback/undo error & stop on error\\
     Tandem computers & fail over - replicas\\
\end{tabular}
\end{itemize}

\end{document}