\documentclass[twoside]{article}
\setlength{\oddsidemargin}{0.25 in}
\setlength{\evensidemargin}{-0.25 in}
\setlength{\topmargin}{-0.6 in}
\setlength{\textwidth}{6.5 in}
\setlength{\textheight}{8.5 in}
\setlength{\headsep}{0.75 in}
\setlength{\parindent}{0 in}
\setlength{\parskip}{0.1 in}

\usepackage{graphicx}
\usepackage{url}

%
% The following commands sets up the lecnum (lecture number)
% counter and make various numbering schemes work relative
% to the lecture number.
%
\newcounter{lecnum}
\renewcommand{\thepage}{\thelecnum-\arabic{page}}
\renewcommand{\thesection}{\thelecnum.\arabic{section}}
\renewcommand{\theequation}{\thelecnum.\arabic{equation}}
\renewcommand{\thefigure}{\thelecnum.\arabic{figure}}
\renewcommand{\thetable}{\thelecnum.\arabic{table}}
\newcommand{\dnl}{\mbox{}\par}

%
% The following macro is used to generate the header.
%
\newcommand{\lecture}[4]{
  \pagestyle{myheadings}
  \thispagestyle{plain}
  \newpage
  \setcounter{lecnum}{#1}
  \setcounter{page}{1}
  \noindent
  \begin{center}
  \framebox{
     \vbox{\vspace{2mm}
   \hbox to 6.28in { {\bf CMPSCI~630~~~ Systems
                       \hfill Fall 2019} }
      \vspace{4mm}
      \hbox to 6.28in { {\Large \hfill Lecture #1  \hfill} }
%       \hbox to 6.28in { {\Large \hfill Lecture #1: #2  \hfill} }
      \vspace{2mm}
      \hbox to 6.28in { {\it Lecturer: #3 \hfill Scribes: #4} }
     \vspace{2mm}}
  }
  \end{center}
  \markboth{Lecture #1: #2}{Lecture #1: #2}
  \vspace*{4mm}
}

%
% Convention for citations is authors' initials followed by the year.
% For example, to cite a paper by Leighton and Maggs you would type
% \cite{LM89}, and to cite a paper by Strassen you would type \cite{S69}.
% (To avoid bibliography problems, for now we redefine the \cite command.)
%
\renewcommand{\cite}[1]{[#1]}

% \input{epsf}

%Use this command for a figure; it puts a figure in wherever you want it.
%usage: \fig{NUMBER}{FIGURE-SIZE}{CAPTION}{FILENAME}
\newcommand{\fig}[4]{
           \vspace{0.2 in}
           \setlength{\epsfxsize}{#2}
           \centerline{\epsfbox{#4}}
           \begin{center}
           Figure \thelecnum.#1:~#3
           \end{center}
   }

% Use these for theorems, lemmas, proofs, etc.
\newtheorem{theorem}{Theorem}[lecnum]
\newtheorem{lemma}[theorem]{Lemma}
\newtheorem{proposition}[theorem]{Proposition}
\newtheorem{claim}[theorem]{Claim}
\newtheorem{corollary}[theorem]{Corollary}
\newtheorem{definition}[theorem]{Definition}
\newenvironment{proof}{{\bf Proof:}}{\hfill\rule{2mm}{2mm}}

% Some useful equation alignment commands, borrowed from TeX
\makeatletter
\def\eqalign#1{\,\vcenter{\openup\jot\m@th
 \ialign{\strut\hfil$\displaystyle{##}$&$\displaystyle{{}##}$\hfil
     \crcr#1\crcr}}\,}
\def\eqalignno#1{\displ@y \tabskip\@centering
 \halign to\displaywidth{\hfil$\displaystyle{##}$\tabskip\z@skip
   &$\displaystyle{{}##}$\hfil\tabskip\@centering
   &\llap{$##$}\tabskip\z@skip\crcr
   #1\crcr}}
\def\leqalignno#1{\displ@y \tabskip\@centering
 \halign to\displaywidth{\hfil$\displaystyle{##}$\tabskip\z@skip
   &$\displaystyle{{}##}$\hfil\tabskip\@centering
   &\kern-\displaywidth\rlap{$##$}\tabskip\displaywidth\crcr
   #1\crcr}}
\makeatother

% **** IF YOU WANT TO DEFINE ADDITIONAL MACROS FOR YOURSELF, PUT THEM HERE:



% Some general latex examples and examples making use of the
% macros follow.

\begin{document}
%FILL IN THE RIGHT INFO.
%\lecture{**LECTURE-NUMBER**}{**DATE**}{**LECTURER**}{**SCRIBE**}
\lecture{16}{November 12}{Emery Berger}{Shashwat Singh}

\section{Failure Oblivious Computing}
Software fault tolerant systems can be used through the use of assert(0), throwing an exception like for eg. x=1/0, x= *0, x=(null)

Latent programmer intent can be specified using exit(0) in UNIX.

Memory safety errors include buffer overflows(spatial), dangling pointers(temporal)

Race condition is specified as if two threads try to compete for the same memory location then it accounts as race condition. 

Emery talks about Hans Boehm designing the android calculator which was designed to give accurate precisions. 

An Undefined behavior can result in a compiler to delete codes.

Acquiring a lock is sometimes expensive so people sometimes dont do it. 

In Hogwild paper there is no concurrency control as it never loses locks. It carefully leads to convergence and gives right result and has higher performance but gives undefined behavior.

Emery talks about the phenomenon in which we don't really notice a significant difference between the colors of the clothes of the characters since humans are not much perceptive towards colors. 

The concept of automatic data structure repair was discussed in which for eg. like in a linkedlist if the backpointer is missing then a back-pointer is added.

Failure oblivious mentions taking a C program and making it a safe C program. If an out of bounds allocation occurs, it is stored in a separate buffer and retrieved later. 

Failstop basically stops the program from running whenever a failure occurs in the program.

Emery told a trick to save the space occupied by backward pointers in doubly linkedlist, in which he told that the backward pointers can be calculated using A XOR C for B, B XOR D for C etc.

Emery talks about why ATC towers are located at a higher elevation, so that when the software fails, the staff can use binoculars to located the aircraft and can use walkie talkie to transmit or receive the messages.

Emery mentions why sometimes reinstalling an app can help in getting rid of corrupted data which was earlier causing the app to fail while running.

Linux has a Out of Memory killer in which it finds if the system is consuming a lot of memory then it exits it.

Fault tolerant systems use the concept of infinite heap semantics.

The Expectation[No of user failures]= p * N

The companies like IBM, Boeing and AT&T are given as examples of N variant systems. It was assumed that the softwares developed by these companies will be independent but was not. 


\end{document}