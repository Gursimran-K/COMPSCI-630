\documentclass[twoside]{article}
\setlength{\oddsidemargin}{0.25 in}
\setlength{\evensidemargin}{-0.25 in}
\setlength{\topmargin}{-0.6 in}
\setlength{\textwidth}{6.5 in}
\setlength{\textheight}{8.5 in}
\setlength{\headsep}{0.75 in}
\setlength{\parindent}{0 in}
\setlength{\parskip}{0.1 in}

\usepackage{graphicx}
\usepackage{url}
\usepackage{amssymb}

%
% The following commands sets up the lecnum (lecture number)
% counter and make various numbering schemes work relative
% to the lecture number.
%
\newcounter{lecnum}
\renewcommand{\thepage}{\thelecnum-\arabic{page}}
\renewcommand{\thesection}{\thelecnum.\arabic{section}}
\renewcommand{\theequation}{\thelecnum.\arabic{equation}}
\renewcommand{\thefigure}{\thelecnum.\arabic{figure}}
\renewcommand{\thetable}{\thelecnum.\arabic{table}}
\newcommand{\dnl}{\mbox{}\par}

%
% The following macro is used to generate the header.
%
\newcommand{\lecture}[4]{
  \pagestyle{myheadings}
  \thispagestyle{plain}
  \newpage
  \setcounter{lecnum}{#1}
  \setcounter{page}{1}
  \noindent
  \begin{center}
  \framebox{
     \vbox{\vspace{2mm}
   \hbox to 6.28in { {\bf CMPSCI~630~~~Systems
                       \hfill Spring 2023} }
      \vspace{4mm}
      \hbox to 6.28in { {\Large \hfill Lecture #1  \hfill} }
%       \hbox to 6.28in { {\Large \hfill Lecture #1: #2  \hfill} }
      \vspace{2mm}
      \hbox to 6.28in { {\it Lecturer: #3 \hfill Scribe: #4} }
     \vspace{2mm}}
  }
  \end{center}
  \markboth{Lecture #1: #2}{Lecture #1: #2}
  \vspace*{4mm}
}

%
% Convention for citations is authors' initials followed by the year.
% For example, to cite a paper by Leighton and Maggs you would type
% \cite{LM89}, and to cite a paper by Strassen you would type \cite{S69}.
% (To avoid bibliography problems, for now we redefine the \cite command.)
%
\renewcommand{\cite}[1]{[#1]}

% \input{epsf}

%Use this command for a figure; it puts a figure in wherever you want it.
%usage: \fig{NUMBER}{FIGURE-SIZE}{CAPTION}{FILENAME}
\newcommand{\fig}[4]{
           \vspace{0.2 in}
           \setlength{\epsfxsize}{#2}
           \centerline{\epsfbox{#4}}
           \begin{center}
           Figure \thelecnum.#1:~#3
           \end{center}
   }

% Use these for theorems, lemmas, proofs, etc.
\newtheorem{theorem}{Theorem}[lecnum]
\newtheorem{lemma}[theorem]{Lemma}
\newtheorem{proposition}[theorem]{Proposition}
\newtheorem{claim}[theorem]{Claim}
\newtheorem{corollary}[theorem]{Corollary}
\newtheorem{definition}[theorem]{Definition}
\newenvironment{proof}{{\bf Proof:}}{\hfill\rule{2mm}{2mm}}

% Some useful equation alignment commands, borrowed from TeX
\makeatletter
\def\eqalign#1{\,\vcenter{\openup\jot\m@th
 \ialign{\strut\hfil$\displaystyle{##}$&$\displaystyle{{}##}$\hfil
     \crcr#1\crcr}}\,}
\def\eqalignno#1{\displ@y \tabskip\@centering
 \halign to\displaywidth{\hfil$\displaystyle{##}$\tabskip\z@skip
   &$\displaystyle{{}##}$\hfil\tabskip\@centering
   &\llap{$##$}\tabskip\z@skip\crcr
   #1\crcr}}
\def\leqalignno#1{\displ@y \tabskip\@centering
 \halign to\displaywidth{\hfil$\displaystyle{##}$\tabskip\z@skip
   &$\displaystyle{{}##}$\hfil\tabskip\@centering
   &\kern-\displaywidth\rlap{$##$}\tabskip\displaywidth\crcr
   #1\crcr}}
\makeatother

% **** IF YOU WANT TO DEFINE ADDITIONAL MACROS FOR YOURSELF, PUT THEM HERE:



% Some general latex examples and examples making use of the
% macros follow.

\begin{document}

%FILL IN THE RIGHT INFO.
%\lecture{**LECTURE-NUMBER**}{**DATE**}{**LECTURER**}{**SCRIBE**}
\lecture{15}{May 2}{Emery Berger}{Bryan McCaffery}

\section{Raid}

\begin{itemize}
  \item RAID stands for Redundant Array of Independent Disks.  
  \item The idea is to store data on multiple disks to increase performance and redundancy.
  \item Normally disks are limited to by writing bandwidth, by increasing the number of disks and striping the data you can get around a n way speed up for n disks.
    \begin{itemize}
    \item Striping is the process of writing parts of the data to multiple different disks. 
    \end{itemize}
  \item As the number of disks increase the probability of any one disk failing also exponentially increases.
  \begin{itemize}
      \item If the probability of any one disk failing is 1 percent, 
        \begin{itemize}
            \item The probability 10 disks do not fail is approximately 90 percent
            \item The probability 100 disks do not fail is around 36.6 percent
            \item Probability any fails p(any) $= 1 - (1-(1-0.01)^{n} $

        \end{itemize}
      \end{itemize}
  \item Max Reliability but minimum parallelism(slowest) is one disk
  \item Max Parallelism(fastest) but minimum reliability is N disks
  \item Levels of Raid
  \begin{itemize}
      \item RAID 1 - Mirroring of Data
      \item RAID 2 - Hamming Code
      \item RAID 3 - Byte level striping, plus level II functionality
      \item RAID 4 - Block level interleaving/striping
      \item RAID 5 - Block level interleaving/striping and check information 
  \end{itemize}
\item Recovery From Failures
\begin{itemize}
    \item  Raid has issues recovering from failures
    \item If a disk totally dies, all information has to be retrieved from replicated disks
    \item There is a high chance of failure as millions of blocks could have to be rewritten 
    \item In many cases, back ups are kept in tape as it is easier to just recover from a backup than replicate data from disks
\end{itemize}
\end{itemize}

\section{Robustness}
\begin{itemize}
    \item SSD's
    \begin{itemize}
        \item SSD's have wear. The crystals can only reshape themselves so many times until they can no longer reshape themselves
        \item They keep track of this and do not put things in the bad blocks of memory
    \end{itemize}
    \item N Variant Systems
    \begin{itemize}
        \item Have three different machines running program that should produce the same output
        \item If two of the machines get same output, and third does not use the value that two machines got
        \item Used by NASA

        \item These machines if on the same spacecraft, can not be truly independent as they are subject to same environmental conditions
    \end{itemize}
    \item Failures are not normally Independent of Each Other
    \begin{itemize}
        \item Paper by Google discovered unusual ways things were not independent
        \item Things as simple as buying all hard drives at the same time from same factory can lead to data centers having multiple failures around the same number of cycles
        \item Contractors hired to right similar programs independent of each other commonly find similar solutions
        
    \end{itemize}
    \item Failure Oblivous Computing
    \begin{itemize}
        \item In the case of a bad address/seg fault put value in map with bad address pointing to value
        \item In the case of read from a bad address, retrieve value from map
    \end{itemize}
\end{itemize}

\end{document}



