\documentclass[twoside]{article}
\setlength{\oddsidemargin}{0.25 in}
\setlength{\evensidemargin}{-0.25 in}
\setlength{\topmargin}{-0.6 in}
\setlength{\textwidth}{6.5 in}
\setlength{\textheight}{8.5 in}
\setlength{\headsep}{0.75 in}
\setlength{\parindent}{0 in}
\setlength{\parskip}{0.1 in}

\usepackage{graphicx}
\usepackage{url}
\usepackage{amssymb}

%
% The following commands sets up the lecnum (lecture number)
% counter and make various numbering schemes work relative
% to the lecture number.
%
\newcounter{lecnum}
\renewcommand{\thepage}{\thelecnum-\arabic{page}}
\renewcommand{\thesection}{\thelecnum.\arabic{section}}
\renewcommand{\theequation}{\thelecnum.\arabic{equation}}
\renewcommand{\thefigure}{\thelecnum.\arabic{figure}}
\renewcommand{\thetable}{\thelecnum.\arabic{table}}
\newcommand{\dnl}{\mbox{}\par}

%
% The following macro is used to generate the header.
%
\newcommand{\lecture}[4]{
  \pagestyle{myheadings}
  \thispagestyle{plain}
  \newpage
  \setcounter{lecnum}{#1}
  \setcounter{page}{1}
  \noindent
  \begin{center}
  \framebox{
     \vbox{\vspace{2mm}
   \hbox to 6.28in { {\bf CMPSCI~630~~~Systems
                       \hfill Spring 2023} }
      \vspace{4mm}
      \hbox to 6.28in { {\Large \hfill Lecture #1  \hfill} }
%       \hbox to 6.28in { {\Large \hfill Lecture #1: #2  \hfill} }
      \vspace{2mm}
      \hbox to 6.28in { {\it Lecturer: #3 \hfill Scribe: #4} }
     \vspace{2mm}}
  }
  \end{center}
  \markboth{Lecture #1: #2}{Lecture #1: #2}
  \vspace*{4mm}
}

%
% Convention for citations is authors' initials followed by the year.
% For example, to cite a paper by Leighton and Maggs you would type
% \cite{LM89}, and to cite a paper by Strassen you would type \cite{S69}.
% (To avoid bibliography problems, for now we redefine the \cite command.)
%
\renewcommand{\cite}[1]{[#1]}

% \input{epsf}

%Use this command for a figure; it puts a figure in wherever you want it.
%usage: \fig{NUMBER}{FIGURE-SIZE}{CAPTION}{FILENAME}
\newcommand{\fig}[4]{
           \vspace{0.2 in}
           \setlength{\epsfxsize}{#2}
           \centerline{\epsfbox{#4}}
           \begin{center}
           Figure \thelecnum.#1:~#3
           \end{center}
   }

% Use these for theorems, lemmas, proofs, etc.
\newtheorem{theorem}{Theorem}[lecnum]
\newtheorem{lemma}[theorem]{Lemma}
\newtheorem{proposition}[theorem]{Proposition}
\newtheorem{claim}[theorem]{Claim}
\newtheorem{corollary}[theorem]{Corollary}
\newtheorem{definition}[theorem]{Definition}
\newenvironment{proof}{{\bf Proof:}}{\hfill\rule{2mm}{2mm}}

% Some useful equation alignment commands, borrowed from TeX
\makeatletter
\def\eqalign#1{\,\vcenter{\openup\jot\m@th
 \ialign{\strut\hfil$\displaystyle{##}$&$\displaystyle{{}##}$\hfil
     \crcr#1\crcr}}\,}
\def\eqalignno#1{\displ@y \tabskip\@centering
 \halign to\displaywidth{\hfil$\displaystyle{##}$\tabskip\z@skip
   &$\displaystyle{{}##}$\hfil\tabskip\@centering
   &\llap{$##$}\tabskip\z@skip\crcr
   #1\crcr}}
\def\leqalignno#1{\displ@y \tabskip\@centering
 \halign to\displaywidth{\hfil$\displaystyle{##}$\tabskip\z@skip
   &$\displaystyle{{}##}$\hfil\tabskip\@centering
   &\kern-\displaywidth\rlap{$##$}\tabskip\displaywidth\crcr
   #1\crcr}}
\makeatother

% **** IF YOU WANT TO DEFINE ADDITIONAL MACROS FOR YOURSELF, PUT THEM HERE:



% Some general latex examples and examples making use of the
% macros follow.

\begin{document}

%FILL IN THE RIGHT INFO.
%\lecture{**LECTURE-NUMBER**}{**DATE**}{**LECTURER**}{**SCRIBE**}
\lecture{8}{March 9}{Emery Berger}{Allison Poh}

\section{Project Details}

Project Title: OUROBOROS (named after snake eating its own tail)
\begin{itemize}
  \item Python makes it easy to see byte code: use dis (disassembler) module
  \item In below example, we may think that if we see foo, we can just replace foo with 42. But this is incorrect since python is a dynamic language.
	\begin{verbatim}
	def foo(n):
	    return 42
	\end{verbatim}
  \item Other dynamics: can change the way operators work (e.g., +); in C++ this is called operator overloading
  \item Take a best effort approach: try to write compiler optimizations; rather than looking for every single one, find ones that will not destroy everything
  \item IR = low level representation (usually looks like a control flow graph)
	\begin{itemize}	
		\item Static Single Assignment (SSA) is a widely used IR that introduced new variables (e.g., x $\rightarrow$ x$\_$prime) and essentially turned imperative language into funcitonal language.
		\item For this project, we will build a src-to-src compiler (i.e., python code $\rightarrow$ do something $\rightarrow$ python code)
	\end{itemize}
  \item Visitor pattern: ast$\_$functionDef( ... )
	\begin{itemize}	
		\item Have a class that, by default, does nothing for nodes
		\item When we have a class that's special, then the visitor will visit it
		\item For this project, we have to use this to process ASTs (or we could alternatively write a function that walks the tree)
	\end{itemize}
  \item First step: write simple visitors just to get the feel for the code (start small and get accustomed to this way of programming)
  \item Project may seem straightforward but there are a lot of gotchas. Juan created over 100 tests, so make a lot of tests and share tests with each other in the class
\end{itemize}

\section{FDIV Bug}

\begin{itemize}
  \item Dividing a float in a certain range with another float in another range caused a substantial error. 
  \item This error was baked into a chip, leading to a recall of intell processors (there are known errors all the time, but they aren't nearly as bad as this one).
  \item Led to verification:
	\begin{itemize}
		  \item 1 bit / 1 bit requires 4 tests
		  \item 32 bits / 32 bits requires $2^64$ tests
		  \item 64 bits / 64 bits requires $2^128$ tests
		  \item Exhausive testing does not work for large state spaces
	\end{itemize}
\end{itemize}

\section{Verification}

\begin{itemize}
  \item Formal verification: prove using math (someones PhD dissertation was on formal verification of FDIV bug)
	\begin{itemize}	
		\item ``Testing can only reveal the presence of bugs but not their absence'' (not exactly true, but justifies formal verificaiton)
	\end{itemize}
  \item Model checking: the in-between of exhaustive testing and formal verification
	\begin{itemize}	
		\item The standard approach is to perform automatic testing up to size $n$, collapsing identical states.
	\end{itemize}
  \item What does TicTok do? What is the formal specificaiton of TickTok?
	\begin{itemize}	
		\item What does it mean to be correct? Well, correct with respect to its specification. But is the specificaiotn correct? Is the meta specificaiton correct? ...
		\item Formal verificaiton is hard since it's hard to specify things.
	\end{itemize}
  \item quicksort example:
	\begin{itemize}	
		\item Requirements of quicksort: needs to terminate and, if the input is a multiset, it needs to output a sequence
		\item Fairly simple specification: $\forall  i, j $ where $ i \geq 0, i \leq n, j \geq 0, j \leq n \therefore i \leq j :$ output[$i$] $\leq$ output[$j$]
		\item But what if sorting strings? What does $\leq$ mean then? And what about empty string? Lowercase vs. uppercase? ... Specification becomes long and, once specification is long, it's just another ``identical program'' that is hard to read.
	\end{itemize}
  \item The Spec Problem: the specification is not useful as it is not ``clearly'' correct
  \item The Oracle Problem: where's the truth? if no oracle, then no truth and specification could be wrong
  \item Total correctness: the program = the specification
  \item Partial correctness: the program properties are a subset of the specification (e.g., program never divides by 0)
\end{itemize}

\section{EPIC}

\begin{itemize}

  \item EPIC: Explicitly Parallel Instruction Computing; created by Intel and HP (originally called VLIW for Very Long Instruction Word)
  \item If all the below are independent, then they can run in parallel:
	\begin{verbatim}
	ADD
	ADD
	MULT
	SUB
	\end{verbatim}
  \item Only make a big chunk if you know that everything can be run in parallel
  \item This shifts the responsibility of parallelism to the compiler ... But there's a big problem as this can't be done (e.g., 18 car lane example)
  \item The plan was to create a new chip, Itanium, that uses EPIC. The chip was an unbelievable failure and was nicknamed Itanic. 
	\begin{itemize}	
		\item Major compatibility problems since it used a totally different instruction set
		\item This is impossible in C because of aliasing of pointers (i.e., pointers $p$ and $q$ can point to the same object)
		\item Cominatorial/state space explosion
	\end{itemize}

\end{itemize}

\end{document}




























