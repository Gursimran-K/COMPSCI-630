\documentclass[twoside]{article}
\setlength{\oddsidemargin}{0.25 in}
\setlength{\evensidemargin}{-0.25 in}
\setlength{\topmargin}{-0.6 in}
\setlength{\textwidth}{6.5 in}
\setlength{\textheight}{8.5 in}
\setlength{\headsep}{0.75 in}
\setlength{\parindent}{0 in}
\setlength{\parskip}{0.1 in}

\usepackage{graphicx}
\usepackage{url}
\usepackage{amsmath}
\usepackage{amssymb}
\usepackage{listings}

%
% The following commands sets up the lecnum (lecture number)
% counter and make various numbering schemes work relative
% to the lecture number.
%
\newcounter{lecnum}
\renewcommand{\thepage}{\thelecnum-\arabic{page}}
\renewcommand{\thesection}{\thelecnum.\arabic{section}}
\renewcommand{\theequation}{\thelecnum.\arabic{equation}}
\renewcommand{\thefigure}{\thelecnum.\arabic{figure}}
\renewcommand{\thetable}{\thelecnum.\arabic{table}}
\newcommand{\dnl}{\mbox{}\par}

%
% The following macro is used to generate the header.
%
\newcommand{\lecture}[4]{
  \pagestyle{myheadings}
  \thispagestyle{plain}
  \newpage
  \setcounter{lecnum}{#1}
  \setcounter{page}{1}
  \noindent
  \begin{center}
  \framebox{
     \vbox{\vspace{2mm}
   \hbox to 6.28in { {\bf COMPSCI~630~~~Systems
                       \hfill Fall 2016} }
      \vspace{4mm}
      \hbox to 6.28in { {\Large \hfill Lecture #1: #2  \hfill} }
      \vspace{2mm}
      \hbox to 6.28in { {\it Lecturer: #3 \hfill Scribe(s): #4} }
     \vspace{2mm}}
  }
  \end{center}
  \markboth{Lecture {#1}: #2}{Lecture {#1}: #2}
  \vspace*{4mm}
}

%
% Convention for citations is authors' initials followed by the year.
% For example, to cite a paper by Leighton and Maggs you would type
% \cite{LM89}, and to cite a paper by Strassen you would type \cite{S69}.
% (To avoid bibliography problems, for now we redefine the \cite command.)
%
\renewcommand{\cite}[1]{[#1]}

% \input{epsf}

%Use this command for a figure; it puts a figure in wherever you want it.
%usage: \fig{NUMBER}{FIGURE-SIZE}{CAPTION}{FILENAME}
\newcommand{\fig}[4]{
           \vspace{0.2 in}
           \setlength{\epsfxsize}{#2}
           \centerline{\epsfbox{#4}}
           \begin{center}
           Figure \thelecnum.#1:~#3
           \end{center}
   }

% Use these for theorems, lemmas, proofs, etc.
\newtheorem{theorem}{Theorem}[lecnum]
\newtheorem{lemma}[theorem]{Lemma}
\newtheorem{proposition}[theorem]{Proposition}
\newtheorem{claim}[theorem]{Claim}
\newtheorem{corollary}[theorem]{Corollary}
\newtheorem{definition}[theorem]{Definition}
\newenvironment{proof}{{\bf Proof:}}{\hfill\rule{2mm}{2mm}}

% Some useful equation alignment commands, borrowed from TeX
\makeatletter
\def\eqalign#1{\,\vcenter{\openup\jot\m@th
 \ialign{\strut\hfil$\displaystyle{##}$&$\displaystyle{{}##}$\hfil
     \crcr#1\crcr}}\,}
\def\eqalignno#1{\displ@y \tabskip\@centering
 \halign to\displaywidth{\hfil$\displaystyle{##}$\tabskip\z@skip
   &$\displaystyle{{}##}$\hfil\tabskip\@centering
   &\llap{$##$}\tabskip\z@skip\crcr
   #1\crcr}}
\def\leqalignno#1{\displ@y \tabskip\@centering
 \halign to\displaywidth{\hfil$\displaystyle{##}$\tabskip\z@skip
   &$\displaystyle{{}##}$\hfil\tabskip\@centering
   &\kern-\displaywidth\rlap{$##$}\tabskip\displaywidth\crcr
   #1\crcr}}
\makeatother

% **** IF YOU WANT TO DEFINE ADDITIONAL MACROS FOR YOURSELF, PUT THEM HERE:



% Some general latex examples and examples making use of the
% macros follow.

\begin{document}

%FILL IN THE RIGHT INFO.
%\lecture{**LECTURE-NUMBER**}{**DATE**}{**LECTURER**}{**SCRIBE**}
\lecture{13}{Fault Tolerance}{Emery Berger}{Emily Herbert}

\section{Admission's Control}
Admission's control is a term used to describe managing load in a system.

\subsection{Load Shedding}
The example given was of a bar:

\textit{Imagine a bar. People enter, they are served, and then they exit. The
load of the bar is determined by a combination of (1) how quickly people arrive
and (2) how quickly they are served.}

This brings us to \textbf{Little's Law}:

\begin{align}
L &= \lambda W \\
\text{average number of customers} &= \text{arrival rate} \cdot \text{service time} \\
E[L] &= E[\lambda] \cdot E[W]
\end{align}

This means that the bar is always in one of two states:

\begin{enumerate}
    \item If the service time is lower than the arrival rate, customers are
    served as quickly as possible and are happy :)
    \item If the service time is higher than the arrival rate, this builds a
    queue. This increases wait times and makes everyone sad :(
\end{enumerate}

This concept introduces \textbf{load shedding}. The idea that turning away
customers can (1) improve Quality-of-Service (QoS) and (2) mitigate against
Denial-of-Service (DoS) attacks.

\subsection{Thread Pools}
Servers use thread pools for admission's control. Have a set number of threads
``at the ready'', so that when one is available it can be used. This is instead
of creating threads on demand.

\section{Cryptography}
Security used to be done through ``security through obscurity''. This relies on
the internals of the system being hidden from the users, and it really isn't
secure at all. Modern security uses a different threat model. It is assumed that
the resources that you want secure (e.g. a secret message) \textit{will} be
intercepted, but security is applied to ensure that the details or content are
unreachable. This is where cryptography comes into play.

Types of cryptography:
\begin{itemize}
    \item Stegonography. A key is used to determine the location of the secret
    things. Codewords.
    \item Caesar Cipher. Symmetric crypto scheme where the key is a mapping of
    letters of the alphabet to other letters. The secret message is encrypted
    with forward mapping and decrypted with the backward mapping. This mapping
    can take two forms:
        \begin{enumerate}
            \item All letters are ``shifted'' by some number. \\
            A B C D E \\
            J K L M N \\
            This is bad because it only takes 26 brute-force attempts to
            decrypt.
            \item All letters are mapped to a random letter. \\
            A B C D E \\
            Z K T Q N \\
            This is better because it takes many brute-force attempts to
            decrypt. But...
        \end{enumerate}
    Those ``many brute-force attempts to decrypt'' can be reduced by
    \textbf{frequency analysis}. Frequency analysis uses common word patterns to
    make guesses at decryption (e.g. a single encrypted letter is likely ``I''
    or ``a'' when decrypted). This brings the random letter Caesar Cipher down
    to a manageable range.
    \item One time pad. Best thing ever. Theoretically unable to be decrypted.
    The message is turned into an array of random numbers, where the max of
    those numbers is the length of the message, and the key is the
    corresponding mapping.
    \item Rotating keys + codebooks. If you have a codebook with a bunch of
    keys, all you have to do is choose a new key each day. This is hard to
    decrypt. Apparently Germany used this and called it Enigma.
\end{itemize}

Known plaintext attacks are when you a priori known what part of the message is
going to say, and you use this information to decrease the huge space of
possible decryption patterns.

\section{Encryption}

The goal of encryption is to make it as if the encrypted message is just
random noise. When something is encrypted to be ``random noise'' the best that
a bad actor can do is brute-force the decryption. So, if we use large encryption
keys, this becomes difficult. A system such as this offers a probabilistic
security gaurentees:

\begin{align}
E[\text{time to decrypt with brute force}] = \frac{1}{2} \cdot \text{key space}
\end{align}

Of course, a bad actor can attain your key. In this case, they will be unable
to do both encryption and decryption. This is a property of asymmetric keys.

The best case is if we can encrypt without being able to decrypt. Thus, we want
some function $F$ that is easy to perform, that has some $F^{-1}$ that is
difficult to perform. ... prime factoring!

All this leads us to the modern day scheme: \textbf{public keys and private
keys}. And also our cast of characters:
\begin{itemize}
    \item Alice and Bob are good actors, and they are attempting to send
    messages back and forth.
    \item Mallory is malicious.
    \item Eve is an evesdropper.
    \item Sybil is the representation of reputation based systems (?)
\end{itemize}

A public key consists of multiple components: $(e, d, n)$, where $e$ is
computed from $d$, and $d$ is relatively prime to $(p-1, q-1)$. Using a
combination of public and private keys, Alice and Bob can achieve:
\begin{align}
\text{Signature} &= D_{bob} (message) -> secret \\
\text{Encrypt} &= E_{alice} (secret) -> secret_{alice} \\
\text{Decrypt} &= D_{alice} (secret_{alice}) -> secret_* \\
\text{now} &= message = E_{bob} (secret_*)
\end{align}

Encrypted messages must be of a certain length, so they are filled up to a
certain size with random numbers.

Who keeps they keys? Some public key infrastructure (PKI). It is a known
condition, so to speak, that there must ultimately be one key origin, from
which all trusted exchanges originate. If this is compromised, the whole
system is compromised.

\section{Fault Tolerance}

\begin{align}
\text{availablity} &= \frac{MTBF}{MTBF + MTTR} \\
MTBF &= \text{mean time between failures} \\
MTTR &= \text{mean time to recover} \\
\end{align}

So to increase fault tolerance, you can either increase the MTBF or reduce the
MTTR. There is a system, ROC, that seeks to reduce MTTR.

Fault tolerance recover can either be fail-safe, fail-stop, or fail-over.
Fail-safe systems rollback or undo the error, fail-stop systems blowup on error,
and fail-over systems rely on replicas to vote on errors.

\end{document}
